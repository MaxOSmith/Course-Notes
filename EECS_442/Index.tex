\documentclass[english, 11pt]{article}
\usepackage{notes}
\usepackage{lipsum}

%Global Course Variables
\newcommand{\myCourseCode}{EECS 442}
\newcommand{\myCourseName}{Computer Vision}
\newcommand{\myProf}{Matthew Johnson-Roberson}
\newcommand{\myTerm}{Fall 2015}

%Headers
\lhead{\myCourseName}
\rhead{\fancyplain{}{\rightmark}} 

%Footers
\cfoot{\thepage}

\begin{document}
\titleHeader{\myCourseCode}{\myCourseName}{\myProf}{\myTerm}

%Document information
\rule[0.5ex]{1\columnwidth}{.5pt}
Contributors: Max Smith
\begin{center}
	Latest revision: \today
\end{center}
\toc
\abstr{Computational methods for the recovery, representation and application of visual information. Topics from image formation, binary images, digital geometry, similarity and dissimilarity detection, matching, curve and surface fitting, constraint propagation relaxation labeling, stereo, shading texture, object representation and recognition, dynamic scene analysis and knowledge based techniques. Hardware, software techniques.}

%----------------------------
%Document Begins
%----------------------------
% Unnamed Examples:

\begin{defn}
	A relative path is a path that is referring to your current directory.
\end{defn}

\begin{rem}
	Recall that double quotes do not protect back quotes.
\end{rem}

\begin{exmp}
	$$2+2=3$$
\end{exmp}

With some names:
\begin{defn}[Relative Path]
    A relative path is a path that is referring to your current directory.
\end{defn}

\begin{rem}[Double Quotes]
    Recall that double quotes do not protect back quotes.
\end{rem}

\begin{exmp}[Addition]
    $$2+2=3$$
\end{exmp}

\section{Introduction and Welcome}
\section{Cameras}
\section{Color}
\section{Light and Shading}
\section{Linear Filtering}
\section{Detectors and Descriptors}
\section{Fitting and Matching}
\section{Recognition}
\subsection{Classifiers}
\subsection{Back Propogation}
\subsection{ConvNets}
\section{Face detection}
\section{Camera Calibration}
\section{Single-View Geometry}
\section{Epipolar Geometry}
\section{Stereo}
\section{Structure from Motion}

\end{document}