\documentclass[english, 11pt]{article}
\usepackage{notes}
\usepackage{lipsum}

%Global Course Variables
\newcommand{\myCourseCode}{EECS 445}
\newcommand{\myCourseName}{Introduction to Machine Learning}
\newcommand{\myProf}{Honglak Lee}
\newcommand{\myTerm}{Fall 2015}

%Headers
\lhead{\myCourseName}
\rhead{\fancyplain{}{\rightmark}} 

%Footers
\cfoot{\thepage}

\begin{document}
\titleHeader{\myCourseCode}{\myCourseName}{\myProf}{\myTerm}

%Document information
\rule[0.5ex]{1\columnwidth}{.5pt}
Contributors: Max Smith
\begin{center}
	Latest revision: \today
\end{center}
\toc
\abstr{Theory and implementation of state-of-the-art machine learning algorithms for large-scale real-world applications. Topics include supervised learning (regression, classification, kernel methods, neural networks, and regularization) and unsupervised learning (clustering, density estimation, and dimensionality reduction).}

%----------------------------
%Document Begins
%----------------------------
% Unnamed Examples:

\begin{defn}
	A relative path is a path that is referring to your current directory.
\end{defn}

\begin{rem}
	Recall that double quotes do not protect back quotes.
\end{rem}

\begin{exmp}
	$$2+2=3$$
\end{exmp}

With some names:
\begin{defn}[Relative Path]
    A relative path is a path that is referring to your current directory.
\end{defn}

\begin{rem}[Double Quotes]
    Recall that double quotes do not protect back quotes.
\end{rem}

\begin{exmp}[Addition]
    $$2+2=3$$
\end{exmp}

\section{Readings}
\subsection{Probability Distributions}
\begin{defn}[Binary Variable]
	Single variable that can take on either 1, or 0; $x\in \{0, 1\}$. We denote $\mu$ ($0\leq\mu\leq 1$) to be the probability that the random binary variable $x=1$
	$$p(x=1|\mu)=\mu$$
	$$p(x=0|\mu)=1-\mu$$
\end{defn}

\begin{defn}[Bernoulli Distribution]
	Probability distribution of the binary variable x, where $\mu$ is the probability $x=1$.
	$$\text{Bern}(x|\mu)=\mu^x(1-\mu)^{1-x}$$
	The distribution has the following properties:
	\begin{itemize}
		\item $\text{E}(x)=\mu$
		\item $\text{Var}(x)=\mu (1-\mu)$
		\item $\mathcal{D}=\{x_1,\ldots ,x_N\} \to p(\mathcal{D} | \mu )=\Pi_{n=1}^{N}p(x_n|\mu)$
		\item Maximum likelihood estimator: $\mu_{ML}=\frac{1}{N}\sum_{n=1}^{N}x_n=\frac{numOfOnes}{sampleSize}$ (aka. sample mean)
	\end{itemize}
\end{defn}

\begin{defn}[Binomial Distribution]
	Distribution of $m$ observations of $x=1$, given a sample size of $N$. 
	$$\text{Bin} (m|N,\mu={\substack{N\\m}}\mu^m (1-\mu )^{N-m}$$\
	\begin{itemize}
		\item $\text{E}(m)=N\mu$
		\item $\text{Var}(m)=N\mu (1-\mu )$
	\end{itemize}
\end{defn}

\subsubsection{The Beta Distribution}
In order to develop a Bayesian treatment for fitting data sets, we will introduce a prior distribution $p(\mu)$.

\begin{itemize}[--]
	\item \textbf{Conjugacy:} when the prior and posterior distributions belong to the same family.
\end{itemize}

\begin{defn}[Beta Distribution]
	$$\text{Beta}(\mu |a,b)=\frac{\Gamma (a+b)}{\Gamma (a)\Gamma (b)}\mu^{a-1} (1-\mu )^{b-1}$$
	Where $\Gamma (x)$ is the gamma function.
	The distribution has the following properties:
	\begin{itemize}
		\item $\text{E}(\mu )=\frac{a}{a+b}$
		\item $\text{Var}(\mu )=\frac{ab}{(a+b)^2 (a+b+1)}$
		\item conjugacy
		\item ${a\to\infty || b\to\infty}\to \text{variance}\\to 0$ 
	\end{itemize}

	Conjugacy can be shown by the distribution by the likelihood function (binomial):
	$$p(\mu |m,l,a,b)\propto \mu^{m+a-1} (1-\mu )^{l+b-1}$$
	Normalized to:
	$$p(\mu |m,l,a,b)= \frac{\Gamma (m+a+l+b)}{\Gamma (m+a)\Gamma (l+b)} \mu^{m+a-1} (1-\mu )^{l+b-1}$$
\end{defn}

\begin{itemize}[--]
	\item \textbf{Hyperparameters:} parameters that control the distribution of the regular parameters.
	\item \textbf{Sequential Approach:} method of learning where you make use of an observation one at a time, or in small batches, and then discard them before the next observatiosn are used. (Can be shown with a Beta, where observing $x=1\to a++, x=0\to b++$, then normalizing)
	\item For a finite data set, the posterior mean for $\mu$ always lies between the prior mean and the maximum likelihood estimate.
	\item A general property of Bayesian learning is when we observe more and more data the uncertainty of the posterior distribution will steadily decrease.
	\item More information and examples of probability distributions can be found in Appendix B of Bishop's `Pattern Recognition and Machine Learning.'
\end{itemize}


\subsection{Linear Models for Regression}
\begin{itemize}[--]
	\item \textbf{Linear Regression:} $y(\mathbf{x}, \mathbf{w})=w_0+w_1 x_1+\ldots +w_D x_D$
	\item Limited on linear function of input variables $x_i$
	\item Extend the model with nonlinear functions, where $\phi_j (x)$ are known as basis functions:
		$$y(\mathbf{x}, \mathbf{w})=w_0 +\sum_{j=1}^{M-1}w_j\phi_j (x)$$
	\item $w_0$ allows for any fixed offset in data, and is known as the \textbf{bias parameter}.
	\item Given a dummy variable $\phi_0 (x)=1$, our model becomes:
		$$y(\mathbf{x}, \mathbf{w})=\sum_{j=0}^{M-1}w_j\phi_j (x)=\mathbf{W}^\mathbf{T} \mathbf{\phi} (x)$$
	\item Functions of this form are called \textbf{linear models} because the function is linear in weight.
\end{itemize}

\subsubsection{Maximum likelihood and least squares}
\begin{itemize}[--]
	\item j
\end{itemize}

% \section{Introduction and Overview}
% \section{Supervised Learning: Regression}
% \section{Supervised Learning: Classification}
% \section{Kernel Methods}
% \section{Regularization and Model Selection}
% \section{Advice on using ML Algorithms}
% \section{Neural Networks}
% \section{Unsupervised Learning}
% \section{Gaussian Process}
% \section{Ensemble Methods}
% \section{Sequence Modeling}
% \section{Learning Theory}

\end{document}