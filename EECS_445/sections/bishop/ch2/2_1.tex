\begin{defn}[Binary Variable]
	Single variable that can take on either 1, or 0. $x\in \{0, 1\}$
\end{defn}

We denote $\mu$ ($0\leq\mu\leq 1$) to be the probability that the random binary variable $x=1$
	$$p(x=1|\mu)=\mu$$
	$$p(x=0|\mu)=1-\mu$$

\begin{defn}[Bernoulli Distribution]
	Probability distribution of the binary variable x, where $\mu$ is the probability $x=1$.
	$$\text{Bern}(x|\mu)=\mu^x(1-\mu)^{1-x}$$
	The distribution has the following properties:
	\begin{itemize}
		\item $\text{E}(x)=\mu$
		\item $\text{Var}(x)=\mu (1-\mu)$
		\item $\mathcal{D}=\{x_1,\ldots ,x_N\} \to p(\mathcal{D} | \mu )=\Pi_{n=1}^{N}p(x_n|\mu)$
		\item Maximum likelihood estimator: $\mu_{ML}=\frac{1}{N}\sum_{n=1}^{N}x_n=\frac{numOfOnes}{sampleSize}$ (aka. sample mean)
	\end{itemize}
\end{defn}

\begin{defn}[Binomial Distribution]
	Distribution of $m$ observations of $x=1$, given a sample size of $N$. 
	$$\text{Bin} (m|N,\mu={\substack{N\\m}}\mu^m (1-\mu )^{N-m}$$\
	\begin{itemize}
		\item $\text{E}(m)=N\mu$
		\item $\text{Var}(m)=N\mu (1-\mu )$
	\end{itemize}
\end{defn}

\subsubsection{The Beta Distribution}
In order to develop a Bayesian treatment for fitting data sets, we will introduce a prior distribution $p(\mu)$.

\begin{defn}[Conjugacy] when the prior and posterior distributions belong to the same family. \end{defn}

\begin{defn}[Beta Distribution]
	$$\text{Beta}(\mu |a,b)=\frac{\Gamma (a+b)}{\Gamma (a)\Gamma (b)}\mu^{a-1} (1-\mu )^{b-1}$$
	Where $\Gamma (x)$ is the gamma function.
	The distribution has the following properties:
	\begin{itemize}
		\item $\text{E}(\mu )=\frac{a}{a+b}$
		\item $\text{Var}(\mu )=\frac{ab}{(a+b)^2 (a+b+1)}$
		\item conjugacy
		\item ${a\to\infty || b\to\infty}\to \text{variance}\\to 0$ 
	\end{itemize}

	Conjugacy can be shown by the distribution by the likelihood function (binomial):
	$$p(\mu |m,l,a,b)\propto \mu^{m+a-1} (1-\mu )^{l+b-1}$$
	Normalized to:
	$$p(\mu |m,l,a,b)= \frac{\Gamma (m+a+l+b)}{\Gamma (m+a)\Gamma (l+b)} \mu^{m+a-1} (1-\mu )^{l+b-1}$$
\end{defn}

\begin{defn}[Hyperparameters] parameters that control the distribution of the regular parameters. \end{defn}

\begin{defn}[Sequential Approach] method of learning where you make use of an observation one at a time, or in small batches, and then discard them before the next observatiosn are used. (Can be shown with a Beta, where observing $x=1\to a++, x=0\to b++$, then normalizing) \end{defn}

\begin{rem} For a finite data set, the posterior mean for $\mu$ always lies between the prior mean and the maximum likelihood estimate.\end{rem}

\begin{rem} A general property of Bayesian learning is when we observe more and more data the uncertainty of the posterior distribution will steadily decrease. \end{rem}

More information and examples of probability distributions can be found in Appendix B of Bishop's `Pattern Recognition and Machine Learning.'