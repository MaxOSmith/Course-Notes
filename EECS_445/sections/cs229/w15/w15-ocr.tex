\subsubsection{Problem Descritpion and Pipeline}
\begin{itemize}[--]
	\item \textbf{Photo OCR}: photo optical character recognition
	\item How do we get computers to understand the text that appears in images
	\item Photo OCR pipeline
	\begin{itemize}[--]
		\item Text detection (bounding box)
		\item Character segmentation (segment each character within box)
		\item Character classification (on each segment)
	\end{itemize}
	\item These `pipelines' are common, where each component may be its own machine learning component (not necessary)
	\item This pipeline can be one of the most important design decision
\end{itemize}

\subsubsection{Sliding Windows}
\begin{itemize}[--]
	\item Supervised leanring for pedestrain detection:
	\begin{itemize}[--]
		\item $x=$ pixels in 82x36 image patches
		\item Positive examples ($y=1$), negative ($y=0$)
		\item Where positive examples are images containing a pedestrian
		\item Train a classifier to recognize pedestrians
	\end{itemize}
	\item To check if if a pedestrian in our image, we will take the same bounding box and slide it across the first row of this size on our image
	\item The amount of pixels you shift the image over in this slide is called the \textbf{step-size/stride}
	\item Now we scale this sliding box for different sized objects of interest
	\item We can apply this concept to detect boxes that contain some sort of character
	\item We can expand our guessed areas, and then draw bounding rectangles over areas that have reasonable resolutions for words
	\item We may also need to train a classifier to classify between letters that are split/partial, to find the individual characters in the words
\end{itemize}

\subsubsection{Getting Lots of Data and Artificial Data}
\begin{itemize}[--]
	\item For character recognition, we can make fake data by taking font from the internet and overlaying it on a random background iamge
	\item You can apply different distortions to get even more sophisticated examples
	\item Another way is by taking already obtained data, and introducing distortions
	\item For another example you can alter audio for speech recognition by overlaying backgrounds
	\item Distortion introduced should be representation of the type of noise/distortions in test set
	\item Usually does not help to add purely random/meaningless noise to your data
	\item Discussion on getting more data:
	\begin{itemize}[--]
		\item Make sure you have a low bias classifier before expending the effort (Plot learning curves). eg. keep increasing the number of features/number of hidden units in neural network until you have a low bias classifier
		\item  ``How much work would it be to get 10x as much data as we currently have?''
		\begin{itemize}[--]
			\item Artificial data synthesis
			\item Collect/label it yourself (no. of hours?)
			\item ``Crowd source'' (eg. amazon mechanical turk)
		\end{itemize}
	\end{itemize}
\end{itemize}

\subsubsection{Ceiling Analysis - What Part of the Pipeline to Work on Next}
\begin{itemize}[--]
	\item  What part of the pipeline should you spend the most time trying to improve?
	\item In order to make decisions on developing the system, it will be helpful to have a single real-number evaluation of the system
	\item Feed perfect inputs into each pipeline stage and see the total improvement via this metric
	\item The larger the jump, the more you should focuso on that segment of the pipeline
\end{itemize}

