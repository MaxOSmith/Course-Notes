\subsection{Default Argument}
\begin{itemize}
	\item One way to solve this problem of duplicate definitions is to use something called a \textit{default argument}
	\item \textbf{Default argument}: make an argument optional by specifying the value to be assigned when not specified (default)
\begin{lstlisting}[style=C++]
IntSet(int capacity = 0);
\end{lstlisting}
\end{itemize}

\subsection{Destructor}
\begin{itemize}
	\item Problem: what happens if we have a local \lstinline[style=C++]{IntSet} inside of a function and the function returns?
\begin{lstlisting}[style=C++]
void foo() {
	IntSet is(3);
}  // foo returns

int main() {
	foo();
}
\end{lstlisting}
	\item Array is stored on the heap, and only the pointer will be destroyed.
	\item To solve this memory leak, we have to arrange to de-allocate the integer array whenever the ``enclosing'' IntSetis destroyed.
	\item We do this with a \textbf{destructor}, the opposite of a constructor
	\item A destructor also has the same name as the class, preceded with a tilde
	\item e.g.,
\begin{lstlisting}[style=C++]
class IntSet{
public:
	//...

	//EFFECTS: destroys this IntSet
	~IntSet();

	//...
};

IntSet::~IntSet() {
	delete[] elts;
}
\end{lstlisting}
	\item Destructors run automatically when an object is destroyed
	\item When does the destructor run?
	\begin{itemize}
		\item Local: out of scope
		\item Global: program end
		\item Dynamic: \textbf{when it is deletec}
	\end{itemize}
\end{itemize}

\subsection{Copy Constructor}
\begin{itemize}
	\item In our original \lstinline[style=C++]{IntSet} when a copy is currently performed the pointer to the array of the original set is copied, so both sets will have pointers point to one array. 
	\item The copy constructor has two tasks
	\begin{enumerate}
		\item Initialize the member variables
		\item \textbf{Copy everything} from the other \lstinline[style=C++]{IntSet}
	\end{enumerate}
	\item e.g.,
\begin{lstlisting}[style=C++]
IntSet::IntSet(constIntSet&other) {
	//1. Initialize member variables
	elts= new int[other.elts_capacity];
	elts_size= other.elts_size;
	elts_capacity= other.elts_capacity;

	//2. Copy everything from the other IntSet
	for (inti= 0; i< other.elts_size; ++i)
		elts[i] = other.elts[i];
}
\end{lstlisting}
	\item The copy constructor we've written follows pointers and copiesthe things they point to, rather than just copying the pointers
	\item This is called a \textbf{deep copy}, as opposed to the default behavior which called a \textbf{shallow copy}
\end{itemize}

\subsection{Destructors and polymorphism}
\begin{itemize}
	\item When you create a object, all the constructors run, starting with the base class
	\item Now, let's see what happens when we mix destructors with polymorphism
	\item e.g.,
\begin{lstlisting}[style=C++]
class Animal {
	virtual void talk() {}
	virtual ~Animal {}
};

class Chicken : public Animal {
public:
	virtual void talk()
		{ cout<< "cluck\n"; }
	virtual ~Chicken {}
};

class Horse : public Animal {
public:
	virtual void talk()
		{ cout<< “neigh\n"; }
};
\end{lstlisting}
	\item When running the following code:
\begin{lstlisting}[style=C++]
Animal *a = ask_user(); //input: “Chicken”
// do something with a
delete a; a=0;
\end{lstlisting}
	\item Since dtoris virtual, correct dtors(\lstinline[style=C++]{~Chicken()}, then \lstinline[style=C++]{~Animal()} ) are selected at runtime.
	\item Destructors run from child class towards parent class!
\end{itemize}