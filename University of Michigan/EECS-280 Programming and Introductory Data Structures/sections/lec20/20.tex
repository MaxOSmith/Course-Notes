\begin{itemize}
	\item An iterator allows you to traverse a container
	\item We’ve seen something like this before with arrays
\begin{lstlisting}[style=C++]
int a[SIZE]; // fill a
for (int i=0; i< SIZE; ++i)
	cout<< a[i] << endl;
\end{lstlisting}
	\item How would you do this for a linked list, like our \lstinline[style=C++]{List} type?
	\item In the end, our iterator will work a lot like using a pointer to traverse an array
\end{itemize}

\subsection{Iterator functions}
\begin{enumerate}
	\item \textbf{Create}: an iterator with a constructor
	\item \textbf{Get}: the \lstinline[style=C++]{T} at the iterator’s current position
	\item \textbf{Move}: the iterator to the next position
	\item \textbf{Compare}: two iterators
\end{enumerate}
\begin{lstlisting}[style=C++]
<template T>
class Iterator {
	Node* node_ptr;
public:
	Iterator(); 							// Create
	T& operator* () const; 					// Get
	Iterator& operator++ (); 				// Move
	bool operator!= (Iterator rhs) const;	// Compare
};
\end{lstlisting}

\subsection{Friends}
\begin{itemize}
	\item Our list also needs to functions \lstinline[style=C++]{begin(), end()} 
	\item \lstinline[style=C++]{begin()}: returns an iterator object pointing to first list position
	\item \lstinline[style=C++]{end()}: returns a default iterator object ``past the end'' position
	\item Our \lstinline[style=C++]{Iterator} needs access to the internal workings of the \lstinline[style=C++]{List} in order to perform these functions
	\item We can solve this by making \lstinline[style=C++]{Iterator} a \textbf{friend} of \lstinline[style=C++]{List}
	\item Friendship is \textbf{given}, not taken
	\item The class declares which classes are its friends
	\item e.g.,
\begin{lstlisting}[style=C++]
class List {
	// ...
	class Iterator {
		// ...
		friend class List;
	};
};
\end{lstlisting}
\end{itemize}

\subsection{Using iterators}
\begin{itemize}
	\item With these functionalities we can now use an iterator to traverse a list:
\begin{lstlisting}[style=C++]
List<int> l;
l.insertFront( 3 );
l.insertFront( 2 );
l.insertFront( 1 );

for (List<int>::Iterator i=l.begin(); i!= l.end(); ++i) {
	cout<< *i<< " ";
}

cout<< endl;
\end{lstlisting}
	\item We have successfully created a function abstraction to access a container
\end{itemize}

\subsection{Iterator invalidation}
\begin{itemize}
	\item Once an iterator is created, if the underlying container is modified, the iterator may become invalid.
	\item If the iterator is invalidated, its behavior is undefined, much like an invalid pointer.
	\item The intuition behind this is that the iterator depends on the representation of the container – if that changes, the iterator is likely to miss an element or return an element that no longer exists.
	\item e.g., (BAD)
\begin{lstlisting}[style=C++]
List<int> l;
l.insertFront( 1 );
List<int>::Iterator i=l.begin();
++i; ++i; // oops, went off the end!
\end{lstlisting}
	\item It’s your responsibility to keep your Iterator within the bounds of the container
\end{itemize}

\subsection{Putting it all together}
\begin{lstlisting}[style=C++]
List<Gorilla*> zoo;
zoo.insertFront(new Gorilla("Colo"));
zoo.insertFront(new Gorilla("Koko"));

for (List<Gorilla*>::Iterator i=zoo.begin(); i!= zoo.end(); ++i)
	cout<< (**i).get_name() << endl;

for (List<Gorilla*>::Iterator i=zoo.begin(); i!= zoo.end(); ++i) {
	delete *i; *i=0;
}

cout<< endl;
\end{lstlisting}