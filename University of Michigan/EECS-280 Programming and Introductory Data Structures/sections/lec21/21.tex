\subsection{First-class Objects}
\begin{itemize}
	\item For any entity in a programming language, we say that the entity is \textbf{first-class} if you can do all four of these things:
	\begin{enumerate}
		\item Create them
		\item Destroy them
		\item Pass them as arguments
		\item Return them as values
	\end{enumerate}
	\item Unlike values, types and functions are notfirst class objects in C++; but, they can sometimes come close.
	\item The \textbf{template mechanism} allows us to pass types as arguments
	\item The \textbf{function pointer mechanism} allows us to pass functions as arguments and return them as results
\end{itemize}

\subsection{Motivation}
\begin{itemize}
	\item \lstinline[style=C++]{any_even, any_odd} are both very similar functions.
	\item We would like a function that is generic, and could perform a more general form of this operation
	\item The generic function signature would like like this:
\begin{lstlisting}[style=C++]
bool any_of(const List<int> &l, bool (*pred)(int));
\end{lstlisting}
\end{itemize}

\subsection{Using function pointers}
\begin{itemize}
	\item \lstinline[style=C++]{pred} is a pointer to a function that takes a single integer argument, returning a boolean result
	\item A \textbf{predicate} is used to control an algorithm
	\item The generic implementation is:
\begin{lstlisting}[style=C++]
bool any_of(const List<int> &l, bool (*pred)(int)) {
	for(List<int>::Iterator i=l.begin(); i!=l.end(); ++i)
		if (pred(*i)) return true;

	//reached end without finding elt
	return false;
}
\end{lstlisting}
	\item We can get the odd elements, by using the appropriate predicate:
\begin{lstlisting}[style=C++]
bool isOdd(intn) {
	return (n%2);
}

List<int> l;
// fill l ...
bool contains_odd= any_of(l, isOdd);
\end{lstlisting}
\end{itemize}

\subsection{Functors}
\begin{itemize}
	\item We can redefine the paranthesis operator () and use a class as a function
	\item However, unlike functions with function pointers, objects can have per-object state, which allows us to specialize on a per-object basis.
	\item e.g.,
\begin{lstlisting}[style=C++]
class GreaterN{
	int limit;
public:
	GreaterN(int limit_in) : limit(limit_in) {}
	bool operator() (int n) {
		return n > limit;
	}
};
\end{lstlisting}
	\item We can use this class to ``generate'' specialized functors:
\begin{lstlisting}[style=C++]
GreaterNg2(2);
GreaterNg6(6);
cout<< g2(4) << endl;
cout<< g6(4) << endl;
\end{lstlisting}
	\item Another use for functors: \textbf{Comparison}: used to define order, takes two elements of the same type and ouputs a \lstinline[style=C++]{bool}
\end{itemize}