\subsection{Structs}
\begin{itemize}
	\item A \lstinline[style=C++]{struct} describes a ``compound object'' that comprises one or more elements, each of independent type
	\item Defines a new type, containing only data
	\item e.g., a triangle:
\begin{lstlisting}[style=C++]
struct Triangle {
	double a, b, c; // Edge lengths
}

int main() {
	Triangle t; // New triangle with undefined edges

	// Initialize each member
	t.a = 3;
	t.b = 4;
	t.c = 5;

	// Or initialize member variables all at once, in order:
	Triangle t2 = {3, 4, 5};

	// Copy
	Triangle t3 = t2;
}
\end{lstlisting}
	\item \lstinline[style=C++]{struct} are passed by value, unlike arrays
	\item This can be a problem with large structs, so pass by reference or pointer
\end{itemize}

\subsection{struct vs. array}
\begin{center}
\begin{tabular}[breaklines=true]{p{5cm}|p{5cm}}
	struct & array \\
	\hline
	{\begin{lstlisting}[style=C++]
struct Triangle{/*...*/};
	 	\end{lstlisting}} &
	{\begin{lstlisting}[style=C++]
double edges[3];
	 	\end{lstlisting}} \\
	Stores group of data & Stores group of data \\
	Heterogeneous types & Homogenous types \\
	Access by name & Acces by index \\
	Default pass-by-value & Default pass-by-reference \\
	Creates a custom type & Does not create a new type \\
\end{tabular}
\end{center}

\subsection{struct vs. class}
\begin{center}
\begin{tabular}{l|l}
	struct & class \\
	Heterogeneous aggregate data type & Heterogeneous aggregate data type \\
	C style & C++ style \\
	Contains only data & Contains data and functions \\
	Undefined by default & Constructors can be used to initialize \\
	All data is accessible & Control of data access \\
\end{tabular}
\end{center}

\subsection{Classes}
\begin{itemize}
\begin{lstlisting}[style=C++]
class Triangle {
	public:
	double a,b,c;//edge lengths

	double area(){ //compute area
		double s = (a+b+c)/2;
		double newArea= sqrt(s*(s-a)*(s-b)*(s-c));
		return newArea;
	}
};
\end{lstlisting}
	\item A \lstinline[style=C++]{class} contains both data and functions
	\item These are called \textbf{member data} and \textbf{member functions}
	\item Because member functions are within the same scope as member data, they have access to the member data directly
	\item \lstinline[style=C++]{public} means members are accessible from outside the \lstinline[style=C++]{class}
	\item From outside scope, access \lstinline[style=C++]{class} members just like a \lstinline[style=C++]{struct}
\begin{lstlisting}[style=C++]
int main() {
	Triangle t;
	t.a=3; t.b=4; t.c=5;
	cout << "area = " << t.area() << endl;

	// Copy
	Triangle t2 = t1;
}
\end{lstlisting}
	\item Calling a member function is just like evaluating a function
\end{itemize}

\subsection{Initialization problem}
\begin{itemize}
	\item Classes, similar to structs, has undefined values upon initialization.
	\item This is a common source of bugs.
	\item A \textbf{constructor} is a member function that has the same name as the class
	\item They run automatically when a new object is defined
	\item It's typically used to initialize member variables
	\item A constructor has the same name as the class, and no return value:
\begin{lstlisting}[style=C++]
class Triangle {
public:
	double a, b, c;
	double area() {/*...*/}

	// Constructor
	Triangle() {a=0; b=0; c=0;}

	// Constructors with inputs for initialization 
	Triangle(double a_in, double b_in, double c_in){
		a = a_in; b = b_in; c = c_in;
	}
}

int main() {
	Triangle t; // Constructor executes automatically
	t.area();
}
\end{lstlisting}
	\item The compiler will select the correct constructor
	\item Two different functions with the same name, but different prototypes is called \textbf{function overloading}.
	\item There are several ways to call the initializer:
\begin{lstlisting}[style=C++]
int main() {
	Triangle t = Triangle(3, 4, 5);
	Triangle t2(3, 4, 5);
}
\end{lstlisting}
\end{itemize}

\subsection{Private member variables}
\begin{itemize}
	\item Declaring variables and functions as \lstinline[style=C++]{private} allows them to be only visible within the scope of that respective class.
\begin{lstlisting}[style=C++]
class Triangle {
private:
	double a, b, c;
public:
	//...
}
\end{lstlisting}
	\item In this example, calling \lstinline[style=C++]{t.a} is no longer possible.
	\item By default, every member of a class is \lstinline[style=C++]{private}
\end{itemize}

\subsection{get and set functions}
\begin{itemize}
	\item A \lstinline[style=C++]{get} function is a \lstinline[style=C++]{public} function that returns a copy of a \lstinline[style=C++]{private} member variable:
\begin{lstlisting}[style=C++]
class Triangle {
	//...
public: 
	// EFFECTS: returns edge a, b, c
	double get_a() {return a; }
	double get_b() {return b; }
	double get_c() {return c; }
}
\end{lstlisting}
	\item A \lstinline[style=C++]{set} function is a \lstinline[style=C++]{public} function that modifies a \lstinline[style=C++]{private} member variable:
\begin{lstlisting}[style=C++]
class Triangle {
	//...
public: 
	// REQUIRES: a,b,care non-negative and form a triangle
	// MODIFIES: a, b, c
	// EFFECTS: sets length of edge a, b, c
	void set_a(double a_in);
	void set_b(double b_in);
	void set_c(double c_in);
}
\end{lstlisting}
	\item \lstinline[style=C++]{set} functions allow you to run extra code when a member variable changes
\end{itemize}

\subsection{struct vs. class}
\begin{center}
\begin{tabular}{l|l}
	struct & class \\
	\hline
	Heterogeneous aggregate data type & Heterogeneous aggregate data type \\
	C style & C++ style \\
	Contains only data & Contains data and functions \\
	Undefined by default & Constructs can be used to initialize \\
	All data is accessible & Control of data access \\
\end{tabular}
\end{center}