\subsection{After calibration}
\img{.8}{sections/lec5/p.png}
\begin{itemize}
	\item Internal parameters $K$ are known
	\item $R, T$ are known - but these can only relate $C$ to the calibraiton rig.
	\item You can't estimate $P_i$ from the single image measurement of $p_i$ because it may be anywhere along the line in space.
	\item We also don't have any information in the image to tell us the scale of anything in the world.
\end{itemize}

\subsection{Recovering structure from a single view}
\begin{itemize}
	\item You can make assumptions to get relative sizes of objects in images
	\item Pick a reference plane in the scene
	\item Pick a reference direction (not parallel to the reference plane) in the scene
	\img{.8}{sections/lec5/ref.png}
	\subsubsection{Geometry}
	\img{.8}{sections/lec5/van.png}
	\item Under perspective projection, parallel lines in three-space project to convering lines in the image plane. The common point of intersection, perhaps at infinity, is called the \textbf{vanishing point}
	\begin{itemize}
		\item projection of a point at infinity (goes infinity far)
		\img{.8}{sections/lec5/vp.png}
		\item Any two parallel lines have the same vanishing point
		\item The ray from $C$ through $v$ point is parallel to the lines
		\item AN image may have more than one vanishing point
	\end{itemize}
	\item Two or more vanishing points from lines known to lie in a single 3D plane establish a \textbf{vanishing line}, which completely determines the orientation of the plane
	\img{.7}{sections/lec5/house.png}
\end{itemize}

\subsection{The Cross Ratio}
\img{.8}{sections/lec5/y.png}
\begin{itemize}
	\item $Y$ is desired height to measure
	\item Compute Y from image measurements
	\begin{itemize}
		\item You'll need more than vanishing points to compute this
	\end{itemize}
	\item \textbf{Projective Invariant}: something that does not change under projective transformations (including perspective projection)
	\img{.8}{sections/lec5/l.png}
	\item The \textbf{cross ratio} of 4 collinear points is projective invariant
	$$\frac{||P_3 - P_1||\cdot ||P_4 - P_2||}{||P_3 - P_2||\cdot ||P_4 - P_1||}$$
	\item You can permute the point ordering and it will remain true.
	\item Often called the fundamental invariant of projective geometry
	\img{.8}{sections/lec5/inf.png}
	\item When you consider the points at $\infty$, and the point where the camera would meet the ground plane $v_z$ you have enough points to use the cross ratio on our camera model.
	\item You need the same points in the world and camera system (can't mix and match)
	\item You must make some assumption to determine the relative sizes, typically assume $L$
\end{itemize}

\subsection{Horizon line}
\img{.7}{sections/lec5/hor.png}
\begin{itemize}
	\item Sets of parallel lines on the same plane lead to collinear vanishing points the line is called the \textbf{horizon}
	\img{.8}{sections/lec5/ex.png}
	\item When trying to recover the structure within the camera reference system, we can check if two linesare parallel or not
	\begin{itemize}
		\item If they do, recognize the horizon line
		\item Measure if hte 2 lines meet the horizon
		\item If they do, they are parallel in 3D
		\item Actual scale of scene is not recovered, only relative distances
	\end{itemize}
\end{itemize}

\subsection{Lines in a 2D plane}
$$ax+by+c=0; l=\begin{bmatrix}
	a\\b\\c
\end{bmatrix}$$
\subsubsection{Intersecting lines}
\img{1}{sections/lec5/intersect.png}
\begin{itemize}
	\item The intersection of lines can be calculated with: 
	$$x=l\times l'$$
\end{itemize}

\subsection{Stereo-view geometry}
\img{1}{sections/lec5/tri.png}
\begin{itemize}
	\item Two camera perspectives allow us to find position of objects through \textbf{triangulation}
	\begin{itemize}
		\item This requires a few knowns, $K_1, K_2, R, T$
	\end{itemize}
	\item Small inaccuracies with knowns can lead to situations where the lines may not intersect 
	\item Instead, we'll find where they came close enough
		$$d^2(x_1, M_1X)+d^2(x_2, M_2X)$$
		$$d:=\text{distance between two lines}$$
\end{itemize}

\subsection{Epipolar Geometry}
\img{.7}{sections/lec5/ep.png}
\begin{itemize}
	\item \textbf{Epipolar plane} (grey): intersections of baseline with image planes
	\item \textbf{Baseline} (orange): projectison of other camera center
	\item \textbf{Epipolar lines} (blue): vanishing points of camera motion direction
	\item This framepoint, allows us to consider if two pointsare related by epipolar geometry
	\item Use epipolar lines to find sharing points will greatly save computation cost as a 2D search becomes 1D
	\subsubsection{Parallel image planes}
	\item Baseline intersects the image plane at infinity
	\item Epipoles are at infinity
	\item Epipolar lines are parallel to x-axis
	\subsubsection{Forward translation}
	\img{1}{sections/lec5/f.png}
	\item When a camera moves forward the lines turn into a spiral
\end{itemize}