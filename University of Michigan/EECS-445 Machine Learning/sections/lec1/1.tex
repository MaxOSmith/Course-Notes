\begin{itemize}
	\item Formal definition: A computer program A is said to learn from experience E with respect to some class of tasks T and performance measure P , if its performance at tasks in T, as measured by P, improves with experience E
	\item Informal definition: Algorithms that improve their prediction performance at some task with experience (or data)
	\item e.g., spam filtering, handwritten digit recognition
	\item \textbf{Training}: given some example data you update parameters of your machine learning algorithm
	\item \textbf{Testing}: evaluating how well your algorithm performs on new data
	\item Machine learning tasks:
	\begin{itemize}
		\item Supervised learning:
		\begin{itemize}
			\item Classification 
			\item Regression
		\end{itemize}
		\item Unsupervised learning:
		\begin{itemize}
			\item Clustering
			\item Density estimation
			\item Dimensionality reduction
		\end{itemize}
		\item Reinforcement Learning
		\begin{itemize}
			\item Learning to act
		\end{itemize}
	\end{itemize}
\end{itemize}

\subsection{Supervised Learning}
\begin{itemize}
	\item Goal:
	\begin{itemize}
		\item Given data X in feature space and the labels Y
		\item Learn to predict Y from X
	\end{itemize}
	\item Labels could be discrete or continuous
	\begin{itemize}
		\item Discrete labels: classification
		\item Continuous labels: regression
	\end{itemize}
	\item Classification:
	\begin{itemize}
		\item Given a feature space (e.g., words in a document)
		\item Predict a label space (e.g., topic of document)
	\end{itemize}
	\item Regression: 
	\begin{itemize}
		\item Given a continuous feature space (e.g., market infromation up to time t)
		\item Predict a label space (e.g., shapre price ``\$24.50'')
	\end{itemize}
\end{itemize}

\subsection{Unsupervised Learning}
\begin{itemize}
	\item Goal:
	\begin{itemize}
		\item Given data X without any labels
		\item Learn the structures of the data
	\end{itemize}
	\item ``Learning without teacher''
	\item Clustering:
	\begin{itemize}
		\item ``Grouping into similar examples''
		\item TODO: Image
	\end{itemize}
	\item Lecture cut early, possibly add skipped here.
\end{itemize}


\subsection{Feature Extraction}
\begin{itemize}
	\item Represent data in terms of vectors
	\begin{itemize}
		\item Featurs are statistics or attributes that describe the data
		\item Practitioners tend to tern this data into a feature space
		\item e.g., For housing data useful features may be: number of rooms, square footage, etc.
	\end{itemize}
	\item You can also consider domain knowledge, namely, use knowledge of how the task work to inject features into the domain
	\begin{itemize}
		\item e.g., for OCR, aspect ratio of tight bounding boxes, existence of of vertical/horizontal strokes
	\end{itemize}
\end{itemize}