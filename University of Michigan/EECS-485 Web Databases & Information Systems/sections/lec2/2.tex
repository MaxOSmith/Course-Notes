\subsection{Networking Basics}
\begin{itemize}
	\item \textbf{Circuit switched}: a dedicated channel is established for the duration of a transmission.
	\begin{itemize}
		\item Good for real-time things like voice and teleconferences
		\item Don't have to packet data
		\item Better latency
		\item No waittime
	\end{itemize}
	\item \textbf{Packet switched}: netwrok in which relatively small units of data called \textbf{packets} are routed through a network for transmission.
	\begin{itemize}
		\item Makes better use of network resources
		\item Can survive destruction of a node in the network
		\item Packets can get lost
	\end{itemize}
\end{itemize}

\subsection{Network Protocol Stack Model}
\begin{center}
\begin{tabular}{c | c c}
	Application & User interaction & HTTP, FTP, SMTP\\
	Presentation & Data representation & XML, Cryptography \\
	Session & DIalogue management & ??? \\\
	Transport & Reliable end-to-end link & TCP \\
	Network & Routing via multiple nodes & IP \\
	Data Link & Physical addressing & Ethernet \\
	Physical & Metal or RF representation & 802.11, Bluetooth
\end{tabular}
\end{center}
\begin{itemize}
	\item IP is best-effor; therefore, packets may get dropped or delayed
	\item TCP is reliable because it guarantees data will get there in order
	\item Open System Interconnection (OSI) Reference Model
	\item 
\end{itemize}

\subsection{HTTP Structure}
\begin{itemize}
	\item Request/response protocol:
	\begin{enumerate}
		\item Client opens TCP connection to server and writes a request
		\item Server response appropriately
		\item COnnection is closed
	\end{enumerate}
	\item Completely stateless
	\begin{itemize}
		\item Each request is treated as brand new
	\end{itemize}
	\item Client requests have several possible forms:
	\begin{itemize}
		\item GET, POST, PUT, DELETE, HEAD, TRACE, CONNECT, OPTIONS
		\item Each has an associated parameter
	\end{itemize}
	\item Even a single page can consist of dozen of HTTP requests
\end{itemize}

\subsection{HTTP Client Algorithm}
\begin{enumerate}
	\item Wati for user to type into browser
	\item Break the URL into host and path
	\item Contact host at port 80, send GET <path> HTTP/1.1
	\item Download result code and bytes
	\item Send content bytes to HTML renderer for drawing onscreen
\end{enumerate}

\subsection{HTTP Server Algorithm}
\begin{enumerate}
	\item HTTP server process (or thread) waits for connection from client
	\item Receives a GET /index.html request
	\item Looks in content directory, computs name /content/index.html
	\item Loads file from disk
	\item Write resposne to clinet: 200 OK, followed by bytes for /content/index.html
\end{enumerate}

\subsection{URL Encoding}
\begin{center}
	http://server:port/path\#fragment?search
\end{center}
\begin{itemize}
	\item Server name translated by DNS look-up: www.umich.edu$\to$135.22.87.1
	\item Path is a file name relative to server root
	\item Fragment is identified at the client, ignored by server
	\item Search string is a general-purpose (set of) parameter(s) that the server can use as it pleases.
\end{itemize}