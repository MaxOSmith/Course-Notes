\subsection{Encryption as a Function}
\begin{itemize}
	\item Plaintest string $s$
	\item Encryption key $K_{enc}$
	\item Decryption key $K_{dec}$
	\item Encrypt $s$ with $K_{enc}$ to obtain ciphertext $K_{enc}(s)$
	\item Decrypt $K_{enc}(s)$ with decryption key $K_{dec}$ to reobtain $s$
	\item $K_{dec}(K_{enc}(s))=s$
	\item Encryption applies a reversible fn to some piece of data, yielding something unreadable
	\item Decryption recovers the original data from the unreadable encryption-output
	\item The encryption/decryption algorithm assumed known; the key is secret
\end{itemize}

\subsection{Substitution Ciphers}
\begin{itemize}
	\item Arbitrary 1:1 mapping of alphabet chars, using a \textbf{substitution table}
	\item All of these are vulnerable to frequency analysis:
	\begin{itemize}
		\item letter
		\item word
		\item common phrases
	\end{itemize}
	\item We could make the substitution table larger (translating $n$-gram, not chars):
	\begin{center}\begin{tabular}{l|l}
		Plaintext & Ciphertext \\
		AAA & QWE \\
		AAB & RTY \\
		AAC & ASD
	\end{tabular}\end{center}
	\item How big is substitution table?
	\begin{itemize}
		\item $A^n$ entries, where A is size of alphabet and $n$ is size of grams
	\end{itemize}
	\item Still vulnerable, but requies more text
\end{itemize}

\subsection{Substitution Rules}
\begin{itemize}
	\item Don't store table explicitly; derive table rows using substitution rule
	\begin{itemize}
		\item e..g, $s \text{XOR} k$, where $k$ is the key
		\item Remember: security level depends on size of key
		\item Key of len $b\geq 2^b$ possible keys
	\end{itemize}
	\item XOR ``flips a bit'' for input bits that correspond to key's ``1''
	\item Encrypted string should ideally show no pattern for frequency analysis attack
	\item What's the right key size?
	\begin{itemize}
		\item Who is trying to break the scheme?
		\item Is that good enough?
	\end{itemize}
\end{itemize}

\subsection{Data Encryption Standard (DES)}
\begin{itemize}
	\item DES is a block cipher with 56-bit key
	\item Data transmitted in 64-bit blocks, each may be coded independently
	\item Triple-DES, breaks up text in 3-56 bit chunks and apply DES to each with different keys
	\img{.7}{sections/lec6/des.png}
\end{itemize}

\subsection{Review of Crypto}
\begin{itemize}
	\item The key is traditional crypto is used to encode the substitution rule
	\begin{itemize}
		\item Needed to encrypt and decrypt
	\end{itemize}
	\item Key distribution is the weak link
	\begin{itemize}
		\item Hard to revoke
		\item Disastrous if ``codebook'' is compromised
		\item Hard to distribute 
		\item Impossible for the Web
	\end{itemize}
\end{itemize}

\subsection{Public-Key Cryptography}
\begin{itemize}
	\item Secure communication without key exchange
	\item Each party has a pair of related keys: \textbf{public} and \textbf{private}
	\begin{itemize}
		\item A public key is published freely
		\item A private key is shared with no one
	\end{itemize}
	\item A message enrypted witho ne can be decrypted with the other
	\item YOu can't derive one from the other (this is critical!)
	\subsubsection{Two Modes}
	\item I encrypt using the public key, and decrypt using the private key
	\img{.8}{sections/lec6/two.png}
	\item Anyone can encrypt; only $S$ can decrypt
	\item Used for data confidentiality
	\item If encrypted using private, anyone with public can decrypt
	\item Used for authenticity
	\item So if someone was able to encrypt, then they must have had private, ensuring no man-in-the-middle
\end{itemize}

\subsection{Trapdoor Functions}
\begin{itemize}
	\item Public key cryptography relies on so-called trapdoor functions
	\begin{itemize}
		\item A fn that is easy to compute, but hard to invert without special info
		\item ``easy'' and ``hard'' meant comutationally
	\end{itemize}
	\item Some poor choices: add, multiply
	\item In practice quite difficult to find good trapdoor rfunctions
	\item Most popular one is related to prime factorizaiton; others possible
	\subsubsection{Prime Factorization}
	\item $n=pq$, where $p$ and $q$ are primes
	\begin{itemize}
		\item Given $p$ and $q$, easy to compute $n$
		\item Given $n$, very hard to find $p$ and $q$
	\end{itemize}
	\subsubsection{Fermat's Little Theorem}
	\item For any prime $p$, and any integer $a$: $a^p = a(\mod p)$
	\subsubsection{Chinese Remainder Theorem}
	\item Consider $x = a_i (\mod p_i), \text{ for }i=1,\ldots, k$
	\item CRT: There's a solution for $x$ if $p_i$ are pairwise relatively prime (i.e., have no common factors greater than 1)
	\item If all $a_i$ are 1, then $x=1(\mod p_i)$
\end{itemize}

\subsection{Cryptography}
\begin{itemize}
	\item We choose two large primes: $p, q$
	\item $n=pq$
	\item Next:
	\begin{itemize}
		\item Set $\lambda = (p-1)(q-1)$
		\item Choose $e$ randomly, such that $e<\lambda$
		\item Choose $d$, such that $de=1(\mod \lambda)$
	\end{itemize}
	\item $n$ and $e$ serve as public key
	\begin{itemize}
		\item $n$ is product of primes $p, q$
	\end{itemize}
	\item $n$ and $d$ serve as private key
	\begin{itemize}
		\item Choosing $d$ requires $e$ and $\lambda$
	\end{itemize}
	\item $\text{encrypt}_{n,e}(m)=m^e (\mod n)=c$
	\item $\text{decrypt}_{n,d}(c)=c^D(\mod n)= m$
	\item Slower than DES because exponential instead of XOR or SHIFTs
	\item Will decryption always work?
	\begin{itemize}
		\item $c^d = m^{de}=m^{k\lambda + 1}$
		\item $m^{k\lambda + 1}=m(m^{(p-1)(q-1)})k$
		\item Recall from Fermat that:
		\begin{itemize}
			\item $m^{p-1}=1(\mod p)$
			\item $m^{q-1}=1(\mod q)$
		\end{itemize}
		\item Which implies:
		\begin{itemize}
			\item $m^{(p-1)(q-1)}=1(\mod p)$
			\item $m^{(p-1)(q-1)}=1(\mod q)$
		\end{itemize}
		\item Use the CRT to combine above 2 equations:
			$$m^{(p-1)(q-1)}=1(\mod n),\text{ where } n=pq$$
		\item $c^d=m(1)^k(\mod n)$
		\item $c^d=m(\mod n)$
		\item Thus, decrypted ciphertext $c=\text{msg}(m)$
	\end{itemize}
	\item Send private shared keys via expensive method, then use cheap method once key is shared. HTTPS does this.
\end{itemize}