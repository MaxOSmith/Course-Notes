\documentclass[11pt,letterpaper]{article}
\usepackage{../../notes}

%Global Course Variables
\newcommand{\myCourseCode}{EECS 381}
\newcommand{\myCourseName}{Objecty-Oriented and Advanced Programming}
\newcommand{\myProf}{David Kieras (Maybe)}
\newcommand{\myTerm}{Fall 2015 (Maybe)}
\newcommand{\myLogo}{../um_seal.png}

%Headers
\lhead{\myCourseName}
\rhead{\fancyplain{}{\rightmark}} 

%Footers
\cfoot{\thepage}

\begin{document}
\titleHeader{\myCourseCode}{\myCourseName}{\myProf}{\myTerm}{\myLogo}

%Document information
\rule[0.5ex]{1\columnwidth}{.5pt}
Contributors: Taku Rusike, Max Smith
\begin{center}
	Latest revision: \today
\end{center}
\toc
\abstr{This elective course introduces advanced concepts and techniques in practical C/C++ programming. We will start with a
quick, but deep, introduction to important topics in C programming, and then the course will emphasize object-oriented programming
with the use of single and multiple inheritance and polymorphism, and using the Standard Library algorithms and containers. Key
ideas in object-oriented analysis and design and common design patterns will be introduced. Programming projects will focus on
learning and using techniques that are valuable for professional practice in developing and extending large scale or high-performance
software relatively easily. In addition to these content goals, an important function of the course is to help students develop good
programming practices, style, and skill, with as much personalized coaching and critique of individual student’s code as possible. In
short, the course is intended for those who want to start becoming outstanding programmers.}

%----------------------------
%Document Begins
%----------------------------

\section{Readings}

\subsection{C concepts: prototypes, headers, linkage}
\subsubsection{Variable Names}
\begin{itemize}[--] %http://en.wikibooks.org/wiki/LaTeX/List_Structures
	\item Internal variables
		\begin{itemize}
			\item Internal variables are local to the function it is declared in. 
			\item They can have a can have a length of 31 characters and can only be comprised of integers and letters and numbers (The underscore counts as a letter). 
		\end{itemize}

	\item External variables
		\begin{itemize}
			\item External variables (globals) are variables that exist for all functions. 
			\item External names can only have a length of 6 characters and one case because they may be used by assemblers and loaders.
		\end{itemize}

	\item Variable naming good practice
		\begin{itemize}
			\item You shouldn't start off variables with an underscore because library routines often use those names. 
			\item It is traditional practice to use lower case for variable names, all uppercase for symbolic constants and a capitalized name for object types. 
		\end{itemize}
\end{itemize}


\subsubsection{Data Types and Sizes}
\begin{description} 
	\item[char] a single byte, 8 bits
	\item[int] an integer, size depends on system
	\item[short] an integer that is no larger than an integer (typically 16 bits)
	\item[long] an integer, size is no smaller tahn an int (typically graeter than or equal to 32 bits)
	\item[float] single-precision floating point
	\item[double] double precision floating point
	\item[unsigned (data type)] respective data type that can only be greater than or equal to 0. Because there are no negative numbers, one can have much more positive numbers ($maxSignedNumber * 2 + 1$) 
\end{description}


\subsubsection{Constants}
\begin{itemize}[--]
	\item An integer constant can be represented like as just an int, hexadecimal, decimal or octal: 1234
	\item A long constant is represented as a short constant with a terminal L or l, or it is a number too big to fit in an int: 123456789l or 123456789L
	\item An unsigned integer constant is represented with a u or U at the end of an integer: 1234u or 1234U
	\item An unsigned long constant is represented as a long but with a ul or UL at the end of the constant: 1234ul or 1234UL
	\item A double constant is represented as a number with a decimal point or as an exponent: 43.1
	\item A float constant is the same representation as a double constant but with a f or F suffix: 43.1F or 43.1f
\end{itemize}


%----------------------------
%Quick Reference Examples
%----------------------------

\end{document}
